%!TEX root = ../report.tex

\begin{document}
\chapter{Notes/Remarks}
\section{Related work - Models}

In this section, we will discuss about the methods available for 3D semantic segmentation.
The discussion include a breif peek into traditional 3D semantic segmentation methods and study of deep learning based 3D point cloud segmentation.

Traditional methods involve a complex features extraction and pass these features to a classficiation algorithm such as Support Vector Machines or Random Forests to classify each point the point cloud.
Various authors developed variety of methods to extract the features from the input point cloud.
Some of these methods include segmentation from edge information \cite{bhanu1986range}, construction of complex graph pyramids \cite{koster}.
3D Hough transforms as in \cite{vosselman20013d} and application of RANSAC \cite{schnabel2007efficient} and \cite{tarsha2007hough}.
These traditional methods are now outdated as DNNs proved to better at feature extraction.

\subsection{Deep learning based 3D semantic segmentation}

\begin{figure}
    \centering
    \includestandalone[width=0.8\linewidth]{images/models_plot}
    \caption{Comparison of 3D semantic segmentation methods performance on SemanticKITTI dataset against the number of parameters. 
             Blue points represent point based methods and red represented projection based methods.}
\end{figure}
\end{document}
