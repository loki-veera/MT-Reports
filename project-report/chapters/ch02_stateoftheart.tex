%!TEX root = ../report.tex

\begin{document}
    \chapter{State of the Art}
    In this chapter, we will discuss about the 3D LiDAR datasets available and made an attempt to classify them based on type of acquisition.
    Also we will discuss about the 3D semantic segmentation models, uncertainty estimation methods and OOD methods available.
    \section{3D LiDAR Datasets}
    %Since the advent of AlexNet \cite{alexnet}, deeplearning has been in the rise. 
    %Deep convolutional neural networks are the desired option for image analysis tasks such as classification, detection and segmentation.
    %These deep convolutional networks are successful especially because of two reasons. 
    %First is the architectures and their paralellisation, allowing them to train millions of images on a GPU.
    %Second reason is the development of huge public benchmark datasets such as Imagenet \cite{imagenet}, Pascal VOC \cite{pascalvoc} and COCO \cite{coco} datasets
    %Although deep convolutional networks are huge success in image analysis tasks, they perform poorly on 3D point cloud data.
    LiDAR is one of the central component in the sensor suite for SLAM system in robotic applications \cite{thrun2006stanley}, \cite{patz2008practical}, \cite{hess20162dSLAM} and autonomous driving \cite{li2016vehicle}.
    3D LiDAR data is preferred because, it can provide the exact replica of 3D geometry of the real world represented in the form of 3D point clouds.
    Because of these rich features and widespread use of LiDAR sensors, tasks such as 3D object detection \cite{zhou2018voxelnet}, \cite{PIXOR} and 3D semantic segmentation \cite{qi2017pointnet++}, \cite{3Dmininet} are becoming more predominant area for research.
    
    In this section, we will discuss about the available 3D LiDAR datasets for 3D semantic segmentation task and classify the datasets based on acquisition methods as in \cite{survey3d}.
    \cite{survey3d} classifies the available public datasets into three classes based on the data acquisition process.
    They are \textit{Sequential}, \textit{Static} and \textit{Synthetic} datasets.
    The data for sequential datasets are collected as frame sequences where mechanical LiDAR is mounted on top of a autonomous driivng platform as in Figure \ref{fig:seq_data_lyft}.
    \begin{figure}[h!]
        \begin{subfigure}{0.45\textwidth}
            \centering
            \includegraphics[scale=0.25]{images/sequential_lyft.png}
            \caption{}
        \end{subfigure}
        \begin{subfigure}{0.45\textwidth}
            \centering
            \includegraphics[scale=0.45]{images/SemanticKitti.pdf}
            \caption{}
        \end{subfigure}
        \caption{Image (a) depicting the LiDAR mounted on a vehicle to collect sequential data for Lyft L5 dataset and 
        (b) depicts an example of sequential data from SemanticKITTI dataset. Images are from \cite{Lyftl5} and \cite{Hu_2020_CVPR_Randla} respectively.}
        \label{fig:seq_data_lyft}
    \end{figure}
    Most of the popular autonomous driving datasets are of sequential type, but these kind of datasets comes with a drawback of sparse points than other datasets.
    
    Static datasets consists of data collected from a stationary view point by a terrestrial laser scanner.
    These kind of datasets capture the static information of the realworld whereas the sequential datasets capture the dynamic movements of the surrounding objects.
    Static datasets find their way in applications such as the urban planning, augmented reality and robotics. 
    Figure \ref{fig:tls} depcits a terrestrial laser scanner used to capture point cloud of an industrial environment.
    \begin{figure}[h!]
        \centering
        \begin{subfigure}{0.45\textwidth}
            \includegraphics[scale=0.45]{images/TLS.jpg}
            \caption{}
        \end{subfigure}
        \begin{subfigure}{0.45\textwidth}
            \centering
            \includegraphics[scale=0.2]{images/sem3d_data/1.pdf}
            \caption{}
        \end{subfigure}

        \caption{Image (a) depicting the terrestrial laser scanner in an industrial environment to collect static data with the laser scanner mounted on a yellow tripod in the left corner of the floor and 
        (b) represents the scene of a static data from Semantic3D dataset. First image is taken from \cite{tls}.}
        \label{fig:tls}
    \end{figure}
    An advantage with the static datasets, are they can produce highly dense point clouds leading to rich 3D geometric representations.
    
    Last type of 3D LiDAR datasets are synthetic datasets. 
    As the name suggests these datasets are generated from the computer simulation. 
    Figure \ref{fig:synthetic} depcits a simulated point cloud in a synthtic dataset called SynthCity.
    Eventhough synthetic datasets can be generated in large scale with cheap cost, they lack the accuracy in detail when compared to the point clouds generated from real world.
    \begin{figure}[h!]
        \centering
        \includegraphics[scale=0.5]{images/synthcity.png}
        \caption{Illustration of a scene in synthetic dataset called SynthCity. Image taken from \cite{griffiths2019synthcity}}
        \label{fig:synthetic}
    \end{figure}
    
    The datasets belonging to the each acquisition type are summed up in  Table \ref{table:3d_lidar_datasets_table}.
    Most of the datasets from the Table \ref{table:3d_lidar_datasets_table} are taken from \cite{survey3d} and also as a part of this study, additional new datasets were added to the list.
    The newly added datasets include DALES \cite{varney2020dales}, ScanObjectNN \cite{scanobejctnn} in static acquisition mode and AIO Drive \cite{Weng2020_AIODrive}, Toronto3D \cite{tan2020toronto3d} are additions in the sequential mode.
    \cite{survey3d} also classifies GTAV \textcolor{red}{\textbf{cite}} dataset as synthetic 3D LiDAR but the corresponding paper doesn't report any LiDAR dataset and proposed only 2D dataset for segmentation.
    The limited number of datasets in 3D LiDAR allowed us to study the characteristics of each individual datasets such as each class, data distribution and features of each point in point cloud. 
    %%%%% It is summarized in Table \textcolor{red}{\textbf{ref}} in Appendix \textcolor{red}{\textbf{chapter number}}
    %%\section{TODO}
    %%\begin{itemize}
    %%     \item[$\bullet$] Explain the table
    %%    \item[$\bullet$] Discuss why SemanticKITTI and Semantic3D are of our interest     
    %%\end{itemize}
    
    \begin{table}[h!]
        %\centering
        \begin{tabular}{c|c|c|c|c|c}%|c}
            \hline
            % acquisition type, dataset, frames, #points, classes, I/O, year
            acquisition mode & dataset & frames & points (in million) & classes & scene type \\  % & pub. year \\
            \hline
            \multirow{7}{*}{static} & Oakland\cite{oakland} & 17 & 1.6 &  44 & outdoor \\ % & 2009 \\ 
                                    & Paris-lille-3D\cite{roynard2018paris} & 3 & 143 & 50 & outdoor \\ % & 2018 \\
                                    & Paris-rue-Madame\cite{paris_rue_madame} & 2 & 20 & 17 & outdoor \\ %& 2014 \\
                                    & S3DIS\cite{Armeni_2016_CVPR_S3DIS} & 5 & 215 & 12 & indoor \\ %& 2016 \\
                                    & ScanObjectNN\cite{scanobejctnn} & - & - & 15 & indoor \\ %& 2019 \\
                                    & Semantic3D\cite{hackel2017semantic3d} & 30 & 4009 & 8 & outdoor \\ %& 2017 \\
                                    & TerraMobilita/IQmulus\cite{TerraMobilita} & 10 & 12 & 15 & outdoor\\ % & 2015 \\
                                    & TUM City Campus\cite{gehrung2017approach_tum_campus} & 631 & 41 & 8 & outdoor\\ % & 2016 \\
                                    & DALES\cite{varney2020dales} & 40 (tiles) & 492 & 8 & outdoor\\ % & 2021\\
            \hline
            \multirow{7}{*}{sequential} & A2D2\cite{geyer2020a2d2} & 41277 & 1238 & 38 & outdoor\\ % & 2020\\
                                        & AIO Drive\cite{Weng2020_AIODrive} & 100& - & 23 & outdoor\\ % & 2020\\
                                        & KITTI-360\cite{Xie_2016_CVPR_KITTI_360} & 100K & 18000 & 19 & outdoor\\ %& 2020\\
                                        & nuScenes-lidarseg\cite{caesar2020nuscenes} & 40000 & 1400 & 32& outdoor\\ % & 2020\\
                                        & PandaSet\cite{PandaSet} & 16000 & 1844 & 37 & outdoor \\ %& 2020\\
                                        & SemanticKITTI\cite{Behley_2019_ICCV} & 43552 & 4549 & 28 & outdoor \\ %& 2019\\
                                        & SemanticPOSS\cite{pan2020semanticposs} & 2988 & 216 & 14 & outdoor \\ %& 2020\\
                                        & Sydney Urban\cite{de2013unsupervised} & 631 & - & 26 & outdoor\\ % & 2013\\
                                        & Toronto-3D\cite{tan2020toronto3d} & 4 & 78.3& 8& outdoor\\ % & 2020\\
    
            \hline
            \multirow{1}{*}{synthetic}  & SynthCity\cite{griffiths2019synthcity} & 75000 & 367.9 & 9 & outdoor \\ %& 2019\\
            \hline
        \end{tabular}
        \caption{3D LiDAR datasets classified based on the acquisition type. Table updated from \cite{survey3d}}
        \label{table:3d_lidar_datasets_table}
    \end{table}
    
    From the available datasets, we chose Semantic3D dataset as in-distribution (ID) training dataset. 
    S3DIS is chosen as out-of-distribution (OOD) dataset as S3DIS is much different from Semantic3D.
    Detailed disussion on datasets were done in Chapter \textcolor{red}{cite here}.

    \section{3D semantic segmentation models}

In this section, we will discuss about the methods available for 3D semantic segmentation.
The discussion include a breif peek into traditional 3D semantic segmentation methods and study of deep learning based 3D point cloud segmentation.

\subsection{Traditional approach}
Traditional methods involve a complex features extraction and pass these features to a classficiation algorithm such as Support Vector Machines or Random Forests to classify each point the point cloud.
Various authors developed variety of methods to extract the features from the input point cloud.
Some of these methods include segmentation from edge information \cite{bhanu1986range}, construction of complex graph pyramids \cite{koster}.
3D Hough transforms as in \cite{vosselman20013d} and application of RANSAC \cite{schnabel2007efficient} and \cite{tarsha2007hough}.
These traditional methods are now outdated as DNNs proved to better at feature extraction.

\subsection{Deep learning approach}
Deep learning based models are efficient at segmentation and can be divided into three types.
Initial type include the point based models where the model directly feeds on the 3D point cloud.
Then the other type include projection based models where the model takes in a projected points into 2D image either a range image or bird's eye view.
Final type of models include the use of graph neural networks.

Let us look breifly into the point based methods. Point based methods majorly utilize fully connected layer or traditional convolutional layers.
The major revolution in Deep learning based 3D semantic segmentation came after the PointNet proposed in \cite{Qi_2017_CVPR_pointnet}.
Pointnet applied Multi Layer Perceptrons (MLP) to extract a global feature vector from the unstructured point cloud and then classifies using the fully connected layers.
Since only Pointnet can extract local features, Pointnet++ proposed in \cite{qi2017pointnet++} applies Pointnet over the point cloud iteratively to extract features at global scale and then classifies.
Eventhough the performance of Pointnet and Pointnet++ are weak, they served as a backbone for new archiectures such as RandLA-Net proposed in \cite{Hu_2020_CVPR_Randla}.
Similar to PointNet, RandLA-Net employs MLP for feature extraction and then these features are passed through attention and then for classification.
More details about RandLA-Net can be found in Section~\ref{sec:meth_randla}.
By the time of writing this thesis, RandLA-Net is one overconfident the best performing model with lower parameters across Static, and Sequential datasets.

The other flavours of point based methods include projection onto a d-dimensional lattice or making the point cloud structured.
SPLATNet \cite{Su_2018_CVPR_splatnet} utilizes the Bilateral Convolutional layers to project data on d-dimensional lattice and the calssified.
Where as LatticeNet proposed in \cite{rosu2019latticenet} projects the data onto sparse lattice, then classified and learns the reprojection to 3D using novel DeformSlice module.
\cite{spvnas} proposed SPVNAS where the point cloud is voxelized then classified and then devoxelized.
Finally \cite{Tatarchenko_2018_CVPR_tangconv} proposed Tangent Convolution similar to standard convolutions but with extra multiplicatives.

In projection based models lets first discuss about the models that project the data onto 2D range image.
First of this kind is SqueezeSeg proposed in \cite{Sequeseseg_2018}. It employs SqueezeNet for 2D segmentation and then reprojected back to 3D using Recurrent Conditional Random Fields.
Since the performance of SqueezeSeg is weak, later versions of SqueezeSeg such as SqueezeSegV2 proposed in \cite{SqueezeSegv2} and SqueezeSegV3 as in \cite{xu2020squeezesegv3} employs Context Aggregation Module and Spatially Adaptive Convolutions respectively to improve performance.
RangeNet proposed in \cite{Milioto2019} even has three versions similar to SqueezeSeg, where first two versions uses DarkNet21 and DarkNet53 model for 2D semantic segmentation.
The final version uses DarkNet53 model with kNN to reproject 2D range image to 3D after segmentation.
Similarly 3DMiniNet \cite{3Dmininet} and KPRNet \cite{kochanov2020kprnet} projects data to range image for segmentation with former using MiniNetV2 as backbone model and later using ResNeXt-101 as backbone.

We now discuss the projection based models which involves bird eye view projection.
These type of models are relatively new to the field but has comparable performance to the existing models.
SalsaNet \cite{salsanet2020} and SalsaNext \cite{SalsaNext_2020} are the two models to employ bird eye view projection as input and both uses ResNet model as backbone.
The later one utilizes Lovasz-Softmax loss along with cross entropy and to the best of our knowledge, this is first model to study uncertainty by modelling SalsaNext as Bayesian Neural Network.
PolarNet \cite{polarnet} and Cylinder3D \cite{zhu2020cylindrical} are recent models to incorporate bird eye view projection into the segmentation pipeline.

The detailed summary of each model discussed here is presented in Table \ref{tab:model_relatedwork} along with number of parameters.
Finally, graph neural networks \cite{dyn_graph_cnn} are also able to produce better segmentation results but this study is limited to standard deep learning models and thus grpah neural networks are not included in this study.


    \section{Uncertainty estimation methods}
    This section will discuss existing methods to estimate uncertainty in deep neural networks.
    Here we divide the existing methods into ensemble methods, Bayesian method and others.
    The methods discussed here primarily estimate epistemic uncertainty. Only test time augmentation and gaussian density models estimate aleatoric uncertainty.

    \cite{matias_uncertainty} argues that there are two kinds of uncertainties, they are epistemic and aleatoric uncertainty.
    Epistemic uncertainty is uncertainty in the model becuase of lack of adequate training data or lack of proper model specification.
    \cite{matias_uncertainty} argues that the epistemic uncertainty is most helpful in OOD detection.
    This is because during the training model parameters trains to a particular distribution of data, must produce high uncertainty estimates when they observe data from different distrbution.
    Aleatoric uncertainty is irreducible and as it is a property of the data like measurement noise.
    An example of aleatoric uncertainty in point cloud can be incorrect data from the LiDAR scanner collected on a foggy day, as fog reflects most of the laser rays creating disturbances in the scan.
    \subsection{Ensemble methods}
    Deep ensembles first proposed in  \cite{lakshminarayanan2016simple} is the most prominently used method for uncertainty estimation.
    They exploit the combinatory power of multiple models.
    Multiple instances of the model are initialized randomly, and the same input is fed into all the instances.
    The random initialization leads to slightly different optimization for each model instance. The final output scores from each model are combined by simple averaging.
    Deep ensembles are known to improve the model's overall performance but come with a cost of computational complexity and resource intensiveness.
    Another advantage of using deep ensembled are the lowest correlation between the model instances as the training of the instances are done differently.
    These lower correlation figures lead to the diverse predictions from each model instance.
    In detailed explantion about deep ensembles can be found in Section~\ref{sec:meth_deepensembles}.
\textcolor{red}{Grammarly checkpoint}

    Because of the higher computational complexity and resource intensiveness, multiple flavours of the deep ensembles are proposed such as deep sub-ensembles \cite{deep_subensembles} and masksembles \cite{masksembles}.
    In deep sub-ensembles, the network is divided into two parts called trunk and head.
    The main idea here is to train multiple instances of the head with the same trunk. For example, a trunk can be feature extraction layers in a deep classification network and the head can be the classification layers.
    Since the major motivation of the subensembles is to improve the training speed the performance lacks minutely when compared to deep ensembles as it is a tradeoff between the computational time and quality of uncertainty.
    Masksembles proposed in \cite{masksembles} are realtively new and are a combination of deep ensembles with MC-Dropout.
    Dropout proposed in \cite{Dropout} includes dropping of random neurons where as here a predefined mask is stated for each layer and only those certain neurons are to be dropped everytime.

    Other methods include snapshot ensembles \cite{snapshot_ensembles} which iterates over the multiple local optima in the optimization landscape using cyclic learning rate. 
    The model parameters are saved at each of the local optima. 
    All of the other flavors of deep ensembles are proposed in order to reduce the effect on time but the memory requirements remain mostly the same except the subensembles.
    The other problem with snapshot ensembles are the optimization landscape in deep neural networks are poorly studied, so there one cannot say with certainty that model saved at two local minima are uncorrelated.
    There also exists neural ensemble search \cite{NAS_Ensembles} where the nerual archiecture search is applied over the ensembles.
    Instead of using various instances of the same model, authors in \cite{NAS_Ensembles} use vaiours instances of various models in the archiecture search space.
    \textcolor{red}{Need refactoring a bit}
    Finally, all the above discussed ensmeble models can be group under sampling based methods, because each image is passed to mulitple model instances.    

    \subsection{Bayesian methods}
    Existing neural networks are trained in maximum likelihood manner resulting in a point estimates for the weights.
    The main idea behind bayesian neural networks are to use a distribution over the network parameters.
    That is instead of single fixed weight tesnor for a layer in neural network, a weight tensor is drawn from the distribution for each forward pass.
    The parameters are estimated for input during training by using bayes rule and expressed in Equation~\ref{eq:bayes_1}.
    \begin{equation}
        p(\theta|x, y) = \frac{p(y|x, \theta)p(\theta)}{p(y|x)} = \frac{p(y|x, \theta)p(\theta)}{\int p(y|x, \theta) p(\theta) d\theta} \label{eq:bayes_1}
    \end{equation}
    Here $\theta$ represents network parameters (weights), $p(\theta)$ represents prior distribution over $\theta$, and x and y represents the input data, in our case point cloud and its corresponding semantic point labels
    During inference, the labels are calculated by bayesian model averaging as given in below Equation~\ref{eq:bayes_2}.
    \begin{equation}
        p(y_t|x_t, x, y) = \int p(y_t|x_t, \theta)p(\theta|x, y)d\theta \label{eq:bayes_2}  
    \end{equation}
    Here $\theta$ represents trained network parameters, x and y represents training set, and $x_t$ and $y_t$ represents the test set.
    The integrals in the Equations~\ref{eq:bayes_1} and \ref{eq:bayes_2} cannont be computed because $\theta$ is the continous space and so iterating over all possible values of $\theta$ are not feasible.
    So to acheive tractable $\theta$, there exists various  approximation methods such as Variational Inference (VI), Laplace approxoimation and sampling methods such as Monte Carlo sampling.

    VI is a approximation method where the posterior probability $p(\theta|x, y)$ is approximated by specific distribtions represented by $q(\theta)$.
    Kullback-Liebler Divergence (KLD) is used as a measure to calculate the difference between the two distributions. 
    Since KLD cannot be optimized directly becuase of posterior distribution, a function called Evidence Lower BOund (ELBO) similar to KLD is proposed.
    \cite{Gaussian_Priors} representes the $q(\theta)$ as a gaussian approximation and \cite{weight_uncertainty} proposed bayes by backprop to extend stochastic VI for non gaussian priors.
    \cite{Non_Gaussian_Priors} provides stochastic variational inference that is ELBO loss over mini-batch of data and also assumed the network parameter priors are to be gaussian.
    \cite{Flipout} makes use of reparameterization trick for reduction in variance in gradients. This lead to reformulation of ELBO loss which made it compatible to standard backpropogation.

    Another widely known example for VI is Monte Carlo Dropout (MC-Dropout), in which the dropout layers are reformanulated as random variables with Bernoulli distribtion.
    and as in \cite{bhandary2020evaluating}, and \cite{gawlikowski2021survey} that training with dropout layers can be formulated as VI and predictive uncertainty can be calculated during inference by applying MC-Dropout during inference.
    The other flavour of MC-Dropout is the random incoming activations to the node are dropped instead of nodes themselves and this methods is called Monte Carlo Dropconnect (MC-Dropconnect) as proposed in \cite{gawlikowski2021survey}.
    An example of MC-Dropout and MC-Dropconnect is depecited in Figure~\ref{fig:Dropout_Connect}.
    \begin{figure}[htbp]
        \begin{subfigure}{0.33\textwidth}
            \centering
            \includegraphics[scale=0.33]{images/BaseNW_SOTA.png}
            \caption{}
        \end{subfigure}
        \begin{subfigure}{0.33\textwidth}
            \centering
            \includegraphics[scale=0.33]{images/Dropout_SOTA.png}
            \caption{}
            \label{fig:SOTA_Dropout}
        \end{subfigure}
        \begin{subfigure}{0.33\textwidth}
            \centering
            \includegraphics[scale=0.33]{images/DConnect_SOTA.png}
            \caption{}
            \label{fig:SOTA_Dconnect}
        \end{subfigure}
        \caption{Figure (a) depicting the connections of the normal fully connected layers with nodes and connections between them,
        (b) representing the MC-Dropout where few nodes are dropped randomly and 
        (c) representing the MC-Dropconnect where the weight connections are dropped randomly. All the images are taken from \cite{UQ_Survey}.}
        \label{fig:Dropout_Connect}
    \end{figure}
    Another method for estimating uncertinty is flipout as proposed in \cite{Flipout}.
    Although flipout was proposed to decorrelate gradients in a minibatch by sampling independent weight perturbations for each example, it is used for variational inference.
    
    \textcolor{red}{Need refactoring a bit}
    Sampling methods also called as Monte Carlo methods (not to be confused to Monte Carlo Dropout or Monte Carlo Dropconnect) estimate uncertainty without any approximation of parametric model.
    The most popular method in the category of sampling methods is Markov Chain Monte Carlo (MCMC) sampling stated in \cite{Bishop_Book}.
    Hamiltonian Monte Carlo (HMC) is an another variant of MCMC method and is considered as gold standard algorithm for Bayesian inference as stated in \cite{HMC}.
    Laplace approximaiton methods make use of second order Tayler series approximation to estimate $p(\theta|x, y)$.
    Uncertianty in the laplace approximation methods is calculated by taking Hessian matrix of $log(p(\theta|x, y))$ and can be applied to existing trained neural networks as in \cite{Laplace_approx}.
    Sampling methods and laplace approimation methods arenot studied because the former is computationally expensive and later suffers from infeasibility to compute Hessian matrix for deep neural networks.
    
    
    
    \section{Out-of-distribution (OOD) detection methods}
    In this section, we will discuss about the existing OOD detection methods for 2D classification task and 2D semantic segmentation task.
    To the best of our knowledge, ours is the first work to study OOD detection for the task of 3D semantic segmentation.

    The most widely used benchmarked datasets used for 2D classficiation dataset are CIFAR-10 vs SVHN \cite{liang2017enhancing_ODIN}, CIFAR-10 vs LSUN \cite{hendrycks2016baseline_MSP}, and MNIST vs not-MNIST \cite{hendrycks2016baseline_MSP}. 
    Most of the proposed methods for OOD detection in classification tasks are threshold based methods which are posthoc methods.
    These methods employ a threshold based detector and does not see the OOD data during the training.
    The baseline method for threshold based methods is proposed in \cite{hendrycks2016baseline_MSP}.
    \cite{hendrycks2016baseline_MSP} uses Maximum Softmax Probability (MSP) and argues that in distribtion dataset have higher softmax score and out of distribtion dataset have lower softmax score and computed as in Equation~\ref{eq:SMSP} where $f_i(x)$ is the output of neural network.
    Since the softmax scores can be overconfident \cite{liang2017enhancing_ODIN} proposed Out of DIstribution detector for Neural networks (ODIN) which makes use of calibrated softmax score by addition of temperature constant to softmax scores and computed as in Equation~\ref{eq:SODIN} where $f_i(x)$ is the output of neural network and $T$ is the temperature constant for calibrated softmax scores.
    In addition to calibrated softmax scores, ODIN also adds noise perturbations to the input making the training adversarial.
    ODIN needs access to the OOD samples because the finetuning of perturbation magnitude is made based on these samples.
    \begin{equation}
        S_{MSP}(x) = max_i \frac{exp(f_i(x))}{\sum^{C}_{j=1}exp(f_j(x))}  \label{eq:SMSP}
    \end{equation}
    \begin{equation}
        S_{ODIN}(x) = max_i \frac{exp(f_i(x)/T)}{\sum^{C}_{j=1}exp(f_j(x)/T)}  \label{eq:SODIN}
    \end{equation}
    %\begin{equation}
    %    \begin{cases}
    %        ID if S(x) > \delta_t \\
    %        OOD otherwise
    %    \end{cases}
    %\end{equation}
    \cite{lee2018simple_mahalanobis} has proposed a threshold based OOD detection method using the Mahalanobis distance as confidence score.
    The mahalanonis distance is calculated for every activation map of each layer of the network and these individual Mahalanobis scores are combined to get confidence score.
    The Mahalanobis distance is calculated between activation map and each class distrbution represented as multivariate guassian.
    \cite{Maha_plus_ODIN} proposed the use of Mahalanobis distance and ODIN combined for OOD detection and argues that this combination yields better detection of OOD data.
    \cite{ReAct} proposed Rectified Activations (ReAct) for OOD detection where in applied to final layer of neural network before softmax application, it supresses the higher activations to a constant thereby limiting the effect of noise. 
    ReAct operation is defined in Equation~\ref{eq:react} as
    \begin{equation}
        h^{-}(x) = ReAct(h(x); c) \text{ and } ReAct(x, c) = min(x, c) \label{eq:react}
    \end{equation} 
    where $h(x)$ is final layer activations, $c$ is the constant and $h^{-}(x)$ is ReAct output of $h(x)$.
    The score from this ReAct function is thresholded to detect the OOD data. 
    Recent developments in threshold based models as proposed in \cite{Energy_OOD_1}, and \cite{Energy_OOD_2} is calculation of energy value based on scalar value from logsumexponential operation over softmax scores.
    These energy scores are expected to be low for the In-Distribution (ID) dataset and vice versa for OOD datasets.

    As the OOD data samples have higher uncertainty over ID samples, this allows us to use the uncertainty estimation methods for OOD detection.
    \cite{lakshminarayanan2016simple} proposes the deep ensembles to estimate epistemic uncertainty and utilizes predictive entropy from deep ensembles to classify ID vs OOD datasets.
    Similarly \cite{JAmersfoot_RBF} proposed an use of radial basis function to calculate the epistemic uncertainty and applied threshold over the epistemic uncertainty to classify OOD data.
    \cite{UOOD_BNN} proposes the use of uncertainty estimated from Bayesian Neural Networks (BNNs) to detect OOD.
    It also argues that the choice of prior weights has a major effect on the OOD performance.
    Also argues that the OOD detection performance using BNNs are only marginally superior compared to other methods.
    \cite{Grad_UOOD} proposes the calculation of gradients during the inference and compute uncertainty using the calculated gradients and argues that this apporach is also comparably effective for OOD detection.
    \cite{UOOD_RL1}, and \cite{UOOD_RL2} applies uncertainty to detect OOD data samples for the task of Deep Reinforcement Learning.
    
    In recent years, OOD detection for task of 2D semantic segmentation is getting in to limelight.
    \cite{pixel_OOD} proposes the adaption of few above discussed OOD detection methods to the task of 2D semantic segmentation with few or no modifications.
    \cite{pixel_OOD} also proposed benchmarked datasets for OOD detection at pixel level and introduced a novel OOD detection evaluation metric called MaxIoU.
    \cite{SemSeg_Entropy1}, and \cite{SemSeg_Entropy2} proposed the use of entropy values for OOD detection in 2D semantic segmentation task.
    \cite{MetaSeg} proposed a small fully connected neural network called \textit{MetaSeg} with inputs being the metrics such as entropy, probability margin, meanIoU and variation ratio to classify ID and OOD samples.
    \cite{uncertianty_distillation} proposed uncertainty distillation to compute uncertainty estimates similar to knowledge distillation and argues these uncertainty esitmates can be efficiently used for OOD detection.

    \section{Summary}
    

\end{document}
